\chapter{CONCLUSÃO}

O estágio realizado na Progolden, como trabalho de conclusão de Curso para obtenção do título de Bacharel em Ciência da Computação, proporcionou a aplicação dos conhecimentos adquiridos no período acadêmico, em um contexto de mercado de trabalho e em um projeto de nível nacional.

O estágio proporcionou conhecimentos, tanto em nível acadêmico quanto em nível profissional. Por se tratar de uma \textit{startup} fundada por alunos e professores do Departamento de Ciência da Computação da Universidade Federal de Lavras, o estágio contribuiu para ter uma visão de como todas as áreas da empresa se relacionam e como a empresa surgiu, a partir de ideias desenvolvidas dentro da própria universidade.

%tirar essa parte do plano de desenvolvimento institucional
A participação no projeto de planejamento estratégico possibilitou experiência de desenvolver um produto, no caso, um software, desde sua concepção até a entrega. 

Os conhecimentos adquiridos na disciplina Engenharia de software ajudaram na realização das atividades de levantamento de requisitos, análise de sistemas e documentações do software.

As disciplinas Algoritmos e estruturas de dados e Banco de dados proporcionaram uma base para o desenvolvimento web e a lógica de programação, além de ajudar o estagiário no entendimento completo do software e na correção de problemas encontrados.

Os conhecimentos da disciplina Interface Homem-máquina, como design centrado no usuário, usabilidade e avaliação heurística foram aplicados durante o estágio, possibilitando o estagiário praticar metodologias que serão importantes no mercado de trabalho.

%ta mais pra médio porte kkk
A participação no projeto trouxe conhecimento para a gerência de projetos, como a divisão de atividades entre os membros e prazos de entrega, e a possibilidade de trabalhar em um projeto utilizando metodologias ágeis.

É comum que, durante as disciplinas, o aluno fique atrelado à parte conceitual e teórica, e muitas das vezes, não sabe ou encontra uma alternativa de como utilizá-las na prática. Como por exemplo, a utilização de versionamento de código, uma abordagem empregada pela maioria das empresas de desenvolvimento de software e que, durante o curso, é visto apenas na teoria, em sala de aula. A prática pôde ser exercida durante a realização do estágio.

%ta falando que é pras universidades ?
O estágio mostrou como o curso de Ciência da Computação pode contribuir na resolução de problemas do cotidiano, no caso, a falta de um software de planejamento estratégico que pudesse agilizar o trabalho de entidades organizacionais e empresas. A partir da problemática, a solução foi construída com foco no usuário e em suas reais necessidades.

Ver como a tecnologia pode influenciar, estimular e facilitar os procedimentos sobre outras áreas do conhecimento é uma motivação tamanha para continuar os estudos na área de Ciência da Computação. 

\section{Trabalhos futuros}

Como trabalho futuro, é importante a realização de testes de usabilidade com usuários reais, para que seja possível encontrar os pontos fracos e ajustá-los de acordo com as necessidades, alinhando a entrega com uma usabilidade eficiente. Com o surgimento de novos módulos e funcionalidades do software, é importante realizar novas avaliações heurísticas.