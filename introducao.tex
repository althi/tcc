\chapter{INTRODUÇÃO}

%TODO Inclua na introdução também do que se trata a empresa e tipo de projeto que você trabalhou (Done)
O objetivo deste documento é apresentar as tarefas realizadas durante o estágio na empresa Progolden Soluções Tecnológicas LTDA, uma startup fundada em 2013 com foco no desenvolvimento de soluções tecnológicas inovadoras. O presente documento descreve conhecimentos obtidos nos estudos de Design Centrado no Usuário, Usabilidade, Experiência do Usuário e \textit{Lean UX} (\textit{User Experience}) durante o estágio, aplicados em um projeto de desenvolvimento de software para planejamento estratégico.

O foco do Design Centrado no Usuário (DCU) é estudar quem são os usuários, quais as suas necessidades e como desenvolver soluções para eles. Segundo a ISO 9241-210 \cite{iso9241}, que descreve processos de projeto centrados no ser humano para sistemas interativos; design centrado no usuário é uma atividade multidisciplinar, que incorpora fatores humanos e conhecimentos ergonômicos e técnicas com o objetivo de aumentar a eficácia e produtividade, melhorar as condições de trabalho humano e neutralizar os possíveis efeitos adversos da utilização na saúde humana, segurança e desempenho. O desenvolvimento do produto, então, teve foco no usuário e como este poderá se beneficiar da melhor forma possível. 

%TODO  Está com transcrição literal, tal qual escrito na ISO? se sim, precisa vir entre aspas

Ainda segundo a ISO 9241-210 \cite{iso9241}, existem quatro atividades de projeto centradas no usuário que precisam ser iniciadas nas primeiras etapas de um projeto, que são:

\begin{itemize}

\item "entender e especificar o contexto de uso";

\item "especificar os requisitos do usuário e da organização";

\item "produzir soluções de design";

\item "avaliar e homologar".

\end{itemize}

Juntamente com o DCU, a metodologia de trabalho do projeto foi o Desenvolvimento Ágil. Atualmente, as equipes enfrentam uma intensa pressão de concorrentes que estão usando técnicas com o desenvolvimento de software ágil, integração contínua e implantação contínua para reduzir radicalmente seus tempos de ciclo \cite{gothelf2013lean}. O manifesto ágil para desenvolvimento de software valoriza os indivíduos, interações, software em funcionamento, colaboração com o cliente, e respostas às mudanças \cite{robbinsd:beck2001agile}. 

Durante o estágio, percebeu-se que, intuitivamente, juntamente com a metodologia ágil aplicada, estava-se trabalhando com a metodologia Lean UX (User Experience). 

O \textit{Lean UX} é definido como uma abordagem para um desenvolvimento de software centrado no usuário, especialmente em startups, criando produtos radicalmente novos. São identificadas três principais influências no \textit{Lean UX}: movimento design thinking, método Learn startup e desenvolvimento de software ágil  \cite{gothelf2013lean}.

O  estágio teve como objetivo aplicar conceitos de design centrado no usuário e usabilidade em um software de planejamento estratégico, capacitando o estagiário para o desenvolvimento de soluções, relizando funções de analista de sistemas, desenvolvedor \textit{front-end}, testador e \textit{user experience researcher}. As principais atividades atribuídas durante o estágio foram:

\begin{itemize}

\item realização de análise de requisitos;

\item especificação de sistemas;

\item desenvolvimento de protótipos;

\item realização de avaliação heurística;

\item realização de testes e correções de erros.

\end{itemize}

O restante desse documento está organizado da seguinte forma: o Capítulo 2 apresenta a descrição do local de realização do estágio.  No Capítulo 3, são apresentadas as técnicas e processos utilizados.  O Capítulo 4 apresenta as atividades desenvolvidas e, finalmente, o Capítulo 5 apresenta as conclusões.
